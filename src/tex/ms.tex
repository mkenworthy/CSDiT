%\documentclass[referee]{aa} % for a referee version
%\documentclass[onecolumn]{aa} % for a paper on 1 column  
%\documentclass[longauth]{aa} % for the long lists of affiliations 
%\documentclass[letter]{aa} % for the letters 
\documentclass{aa}

\usepackage{txfonts}
\usepackage{natbib}

\usepackage{graphicx}

\usepackage{color}
\usepackage{hyperref}
\hypersetup{colorlinks=true,allcolors=[rgb]{0,0,0.8}}


\usepackage{showyourwork}

% the three lines suppress the hyperref 'link empty' warnings
% explanation at: https://tex.stackexchange.com/questions/345764/journal-class-shows-package-hyperref-warning-suppressing-link-with-empty-targe
\makeatletter
\renewcommand*\aa@pageof{, page \thepage{} of \pageref*{LastPage}}
\makeatother

\usepackage{xspace}

\newcommand{\ktwo}{\textit{K2}}
\newcommand{\kms}{km~s$^{-1}$\xspace}
\newcommand{\ms}{m~s$^{-1}$}
\newcommand{\gcc}{g~cm$^{-3}$}
\newcommand{\masyr}{mas~yr$^{-1}$}
\newcommand{\err}{\textit{$\pm$}}
\newcommand{\teff}{$T_\mathrm{eff}$}
\newcommand{\msun}{$M_\odot$}
\newcommand{\rsun}{$R_\odot$}
\newcommand{\lsun}{$L_\odot$}
\newcommand{\rhosun}{$\rho_\odot$}
\newcommand{\mstar}{$M_*$}
\newcommand{\rstar}{$R_*$}
\newcommand{\lstar}{$L_*$}
\newcommand{\rearth}{$R_\oplus$}
\newcommand{\vrad}{$v_{R}$}
\newcommand{\pmra}{$\mu_{\alpha}$}
\newcommand{\pmdec}{$\mu_{\delta}$}

\newcommand{\rhostar}{$\rho_*$}
\newcommand{\mjup}{$M_\mathrm{Jup}$}
\newcommand{\galex}{\textit{GALEX}}
\newcommand{\gaia}{\textit{Gaia}}
\newcommand{\kepler}{\textit{Kepler}}
\newcommand{\spitzer}{\textit{Spitzer}}
\newcommand{\ktwosc}{\textsc{k2sc}}
\newcommand{\ktwosff}{\textsc{k2sff}}
\newcommand{\hipparcos}{\textit{Hipparcos}}
\newcommand{\tess}{\textit{TESS}}
\newcommand{\emcee}{\textsc{emcee}}
\newcommand{\python}{\textsc{python}}


\begin{document} 

   \title{Disks in Transit}

   \author{M. Kenworthy
          \inst{1}
          }

   \institute{Leiden Observatory, University of Leiden,
   PO Box 9513, 2300 RA Leiden, The Netherlands\\
   \email{kenworthy@strw.leidenuniv.nl}
  }
   \date{Received XXXX; accepted XXXX}

% \abstract{}{}{}{}{} 
% 5 {} token are mandatory
 
  \abstract
  % context heading (optional)
  % {} leave it empty if necessary  
   {We review the literature on disks that transit stars and understand how they can be analyzed.}
  % aims heading (mandatory)
   {}
  % methods heading (mandatory)
   {}
  % results heading (mandatory)
   {}
  % conclusions heading (optional), leave it empty if necessary 
   {}

   \keywords{giant planet formation --
                $\kappa$-mechanism --
                stability of gas spheres
               }

   \maketitle
%
%-------------------------------------------------------------------

   \section{Introduction}

Terrestrial planets are thought to be built up by the quasi-periodic accretion of planetary embryos that generate a significant amount of ejected material.

lets test this: \citep{Luger2021} and Figure~\ref{fig:asassn21js}.


\begin{figure*}
   \begin{centering}
   \includegraphics[width=\textwidth]{figures/asassn-21js.pdf}
      \caption{ASASSN-21js light curve. Normalised in both bands $g$ and $V$.
              }
              \label{fig:asassn21js}
              \script{plot_asassn-21js.py}
              \end{centering}
       \end{figure*}

\begin{figure*}
   \begin{centering}
   \includegraphics[width=\textwidth]{figures/asassn-21nn.pdf}
      \caption{ASASSN-21nn light curve. Normalised in both bands $g$ and $V$.
              }
              \label{fig:asassn21nn}
              \script{plot_asassn-21nn.py}
              \end{centering}
       \end{figure*}


\begin{figure*}
   \begin{centering}
   \includegraphics[width=\textwidth]{figures/J1706.pdf}
      \caption{J1706 light curve. Normalised in both bands $g$ and $V$.
              }
              \label{fig:J1706}
              \script{plot_J1706.py}
              \end{centering}
       \end{figure*}


\begin{figure*}
   \begin{centering}
   \includegraphics[width=\textwidth]{figures/asassn-21sa.pdf}
      \caption{ASASSN-21sa light curve. Normalised in both bands $g$ and $V$.
              }
              \label{fig:asassn21sa}
              \script{plot_asassn-21sa.py}
              \end{centering}
       \end{figure*}


\begin{figure*}
   \begin{centering}
   \includegraphics[width=\textwidth]{figures/J1816.pdf}
      \caption{J1816 light curve. Normalised in both bands $g$ and $V$.
              }
              \label{fig:J1816}
              \script{plot_J1816.py}
              \end{centering}
       \end{figure*}
%------------------------
\section{Conclusions}\label{sec:conclusion}

   \begin{enumerate}
      \item There was a collision between planetoids towards ASASSN-21qj which generated a debris cloud.
      \item The cloud moved in front of the star, and we have a fresh measure of the debris from a collision.
   \end{enumerate}

\begin{acknowledgements}
Thank matplotlib

\end{acknowledgements}

\bibliographystyle{aa}
\bibliography{bib}

\end{document}

