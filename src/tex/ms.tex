%\documentclass[referee]{aa} % for a referee version
%\documentclass[onecolumn]{aa} % for a paper on 1 column  
%\documentclass[longauth]{aa} % for the long lists of affiliations 
%\documentclass[letter]{aa} % for the letters 
\documentclass{aa}

\usepackage{txfonts}
\usepackage{natbib}

\usepackage{graphicx}

\usepackage{color}
\usepackage{hyperref}
\hypersetup{colorlinks=true,allcolors=[rgb]{0,0,0.8}}


\usepackage{showyourwork}

% the three lines suppress the hyperref 'link empty' warnings
% explanation at: https://tex.stackexchange.com/questions/345764/journal-class-shows-package-hyperref-warning-suppressing-link-with-empty-targe
\makeatletter
\renewcommand*\aa@pageof{, page \thepage{} of \pageref*{LastPage}}
\makeatother

\usepackage{xspace}

\newcommand{\ktwo}{\textit{K2}}
\newcommand{\kms}{km~s$^{-1}$\xspace}
\newcommand{\ms}{m~s$^{-1}$}
\newcommand{\gcc}{g~cm$^{-3}$}
\newcommand{\masyr}{mas~yr$^{-1}$}
\newcommand{\err}{\textit{$\pm$}}
\newcommand{\teff}{$T_\mathrm{eff}$}
\newcommand{\msun}{$M_\odot$}
\newcommand{\rsun}{$R_\odot$}
\newcommand{\lsun}{$L_\odot$}
\newcommand{\rhosun}{$\rho_\odot$}
\newcommand{\mstar}{$M_*$}
\newcommand{\rstar}{$R_*$}
\newcommand{\lstar}{$L_*$}
\newcommand{\rearth}{$R_\oplus$}
\newcommand{\vrad}{$v_{R}$}
\newcommand{\pmra}{$\mu_{\alpha}$}
\newcommand{\pmdec}{$\mu_{\delta}$}

\newcommand{\rhostar}{$\rho_*$}
\newcommand{\mjup}{$M_\mathrm{Jup}$}
\newcommand{\galex}{\textit{GALEX}}
\newcommand{\gaia}{\textit{Gaia}}
\newcommand{\kepler}{\textit{Kepler}}
\newcommand{\spitzer}{\textit{Spitzer}}
\newcommand{\ktwosc}{\textsc{k2sc}}
\newcommand{\ktwosff}{\textsc{k2sff}}
\newcommand{\hipparcos}{\textit{Hipparcos}}
\newcommand{\tess}{\textit{TESS}}
\newcommand{\emcee}{\textsc{emcee}}
\newcommand{\python}{\textsc{python}}


\begin{document} 

   \title{The eclipse of ASASSN-21qj}

   \author{M. Kenworthy
          \inst{1}
          \and
          Arttu Sainio
          \inst{2}
          }

   \institute{Leiden Observatory, University of Leiden,
   PO Box 9513, 2300 RA Leiden, The Netherlands\\
   \email{kenworthy@strw.leidenuniv.nl}
         \and
             Arttu's address ...\\
    \and
    JPL
    \and
    Caltech/IPAC, 1200 E California Blvd, Mail Code 100-22, Pasadena, CA 91125, USA}

   \date{Received XXXX; accepted XXXX}

% \abstract{}{}{}{}{} 
% 5 {} token are mandatory
 
  \abstract
  % context heading (optional)
  % {} leave it empty if necessary  
   {Collisions occur between planetessimals that generate debris disks through collisional cascades.}
  % aims heading (mandatory)
   {We analyze the dust and size distribution of the eclipse seen towards ASASSN-21qj.}
  % methods heading (mandatory)
   {Fit the light curve from three different colours to determine the particle size and distribution.}
  % results heading (mandatory)
   {The eclipse is coloured, indicating dust.
   %
   The dust has a lower limit mass of XXXX Earths, eclipse has a duration of XXXX days.}
  % conclusions heading (optional), leave it empty if necessary 
   {}

   \keywords{giant planet formation --
                $\kappa$-mechanism --
                stability of gas spheres
               }

   \maketitle
%
%-------------------------------------------------------------------

   \section{Introduction}

Terrestrial planets are thought to be built up by the quasi-periodic accretion of planetary embryos that generate a significant amount of ejected material.

lets test this: \citep{Luger2021}
%
%------------------------
\section{Conclusions}\label{sec:conclusion}

   \begin{enumerate}
      \item There was a collision between planetoids towards ASASSN-21qj which generated a debris cloud.
      \item The cloud moved in front of the star, and we have a fresh measure of the debris from a collision.
   \end{enumerate}

\begin{acknowledgements}
Thank matplotlib

\end{acknowledgements}

\bibliographystyle{aa}
\bibliography{bib}

\end{document}

% template.tex, dated April 5 2013
% This is a template file for Annual Reviews 1 column Journals
%
% Compilation using ar-1col.cls' - version 1.0, Aptara Inc.
% (c) 2013 AR
%
% Steps to compile: latex latex latex
%
% For tracking purposes => this is v1.0 - Apr. 2013

\documentclass[letterpaper]{ar-1col}
\usepackage{showyourwork}
\usepackage[letterpaper]{geometry}

\usepackage{natbib}
\usepackage{amsmath}
\usepackage{color}
\usepackage{hyperref}
\hypersetup{hidelinks}

\usepackage{graphbox}
%\newcommand{\suz}[1]{\textcolor{magenta}{#1}}
\newcommand{\dan}[1]{\textcolor{green}{#1}}

\setcounter{secnumdepth}{4}
\usepackage{url}

% Metadata Information
\jname{Xxxx. Xxx. Xxx. Xxx.}
\jvol{AA}
\jyear{202x}
\doi{10.1146/((please add article doi))}

% autoref formatting
\def\sectionautorefname{Section}
\let\subsectionautorefname\sectionautorefname
\let\subsubsectionautorefname\sectionautorefname

% macros
\newcommand{\apjl}{Astrophysical Journal Letters}
\newcommand{\aj}{Astronomical Journal}
\newcommand{\apj}{Astrophysical Journal}
\newcommand{\apjs}{Astrophysical Journal Supplement}
\newcommand{\pasp}{Publications of the Astronomical Society of the Pacific}
\newcommand{\jgr}{Journal of Geophysical Research}
\newcommand{\aap}{Astronomy and Astrophysics}
\newcommand{\mnras}{Monthly Notices of the Royal Astronomical Society}
\newcommand{\actaa}{Acta Astronomica}
\newcommand{\nat}{Nature}
\newcommand{\prl}{Physical Review Letters}

% Symbols
\newcommand{\ydata}{\ensuremath{\boldsymbol{y}}}
\newcommand{\hyperparams}{\ensuremath{\boldsymbol{\phi}}}
\newcommand{\meanparams}{\ensuremath{\boldsymbol{\theta}}}
\newcommand{\dt}{\ensuremath{\tau}}
\newcommand{\amplitude}{\ensuremath{\alpha}}
\newcommand{\lengthscale}{\ensuremath{\lambda}}

\DeclareMathOperator*{\argmax}{arg\,max}

% Document starts
\begin{document}

% Page header
\markboth{Kenworthy}{Disks in Transit}

% Title
\title{Circumsecondary Disks in Transit}

%Authors, affiliations address.
\author{Matthew A. Kenworthy$^1$
  \affil{$^1$Leiden Observatory, Leiden University, P.O. Box 9513, 2300 RA Leiden, The Netherlands; email: kenworthy@strw.leidenuniv.nl}
 }

%Abstract
\begin{abstract}
  All the possible circumplanetary/circumsecondary disk light curves to date. 
  %
  I'm writing it in the ARA\&A template because it looks pretty and I need all the dopamine hits I can get.
\end{abstract}

%Keywords, etc.
\begin{keywords}
  keywords, separated by comma, no full stop, lowercase
\end{keywords}
\maketitle

%Table of Contents
\tableofcontents

\section{INTRODUCTION}
\label{sec:intro}

\begin{armarginnote}[]
  \entry{CSD}{Circumstellar Disks}
  \entry{CPD}{Circumplanetary Disks}
\end{armarginnote}

Recent direct detection of CPD around PDS 70b has energised the field of detections of disks around planets.
%
These disks are the process by which material from the circumstellar disk passes onto the surface of a forming planet, and once the CSD reservoir of material is depleted, leads to the formation of moons in a coplanar configuration around the gas giant planet.

\subsection{Brief history}

Disks are ubiquitous, circumstellar disks, the awesome image of a disk transiting in front of a star with the IOTA interferometer.

Allows determination of the matgerial within the disk for stars.

Dust, 



\subsection{Motivating examples}
\label{sec:sim_examples}

\begin{table}[ht]
  \caption{Some kernel functions commonly used in the astrophysics literature \dan{add the rational quadratic kernel}}
  \label{tab:kernels}
  \begin{center}
    \begin{tabular}{@{}l|l@{}}
      \hline
      Name                           &  Representation$^{\rm a}$                                    \\
      \hline
      Constant kernel                & $\amplitude^2$                                                     \\
      Squared Exponential$^{\rm b}$  & $e^{-(\dt/\lengthscale)^2/2}$                                              \\
      Exponential$^{\rm c}$          & $e^{-\dt/\lengthscale}$                                                  \\
      Mat\'ern-3/2                   & $\left(1 + \sqrt{3}\dt/\lengthscale\right)e^{-\sqrt{3}\dt/\lengthscale}$             \\
      Mat\'ern-5/2                   & $\left(1 + \sqrt{5}\dt/\lengthscale +5(\dt/\lengthscale)^2/3\right)e^{-\sqrt{5}\dt/\lengthscale}$  \\
      Cosine                         & $\cos 2\pi\dt/\lengthscale$                                                  \\
      Exponential Sine Squared       & $\exp\left(-\gamma\sin^2\pi\dt/\lengthscale\right)$                      \\
      Stochastic Harmonic Oscillator$^{\rm d}$ & $\cos\left(\sqrt{1-\beta^2}\frac{\dt}{\lengthscale}\right) + \frac{\beta}{\sqrt{1-\beta^2}}\sin\left(\sqrt{1-\beta^2}\frac{\dt}{\lengthscale}\right)$                          \\
      \hline
    \end{tabular}
  \end{center}
  \begin{tabnote}
    $^{\rm a}$in each case, $\dt$ is defined as $\dt = \left|t_i - t_j\right|$, and Greek letters indicate hyper-parameters;
    $^{\rm b}$``radial basis function'';
    $^{\rm c}$``Ornstein-Uhlenbeck'', ``damped random walk'' or ``Mat\'ern-1/2'';
    $^{\rm d}$\citet{celerite}.
  \end{tabnote}
\end{table}


% \section{FIRST-LEVEL HEADING}
% This is dummy text.
% % Heading 2
% \subsection{Second-Level Heading}
% This is dummy text. This is dummy text. This is dummy text. This is dummy text.

% % Heading 3
% \subsubsection{Third-Level Heading}
% This is dummy text. This is dummy text. This is dummy text. This is dummy text.

% % Heading 4
% \paragraph{Fourth-Level Heading} Fourth-level headings are placed as part of the paragraph.

% %Example of a Figure
% \section{ELEMENTS\ OF\ THE\ MANUSCRIPT}
% \subsection{Figures}Figures should be cited in the main text in chronological order. This is dummy text with a citation to the first figure (\textbf{Figure \ref{fig1}}). Citations to \textbf{Figure \ref{fig1}} (and other figures) will be bold.

% \begin{figure}[ht]
%   \includegraphics[width=3in]{SampleFigure}
%   \caption{Figure caption with descriptions of parts a and b}
%   \label{fig1}
% \end{figure}

% % Example of a Table
% \subsection{Tables} Tables should also be cited in the main text in chronological order (\textbf {Table \ref{tab1}}).

% \begin{table}[ht]
%   \tabcolsep7.5pt
%   \caption{Table caption}
%   \label{tab1}
%   \begin{center}
%     \begin{tabular}{@{}l|c|c|c|c@{}}
%       \hline
%       Head 1              &          &                   &         & Head 5    \\
%       {(}units)$^{\rm a}$ & Head 2   & Head 3            & Head 4  & {(}units) \\
%       \hline
%       Column 1            & Column 2 & Column3$^{\rm b}$ & Column4 & Column    \\
%       Column 1            & Column 2 & Column3           & Column4 & Column    \\
%       Column 1            & Column 2 & Column3           & Column4 & Column    \\
%       Column 1            & Column 2 & Column3           & Column4 & Column    \\
%       \hline
%     \end{tabular}
%   \end{center}
%   \begin{tabnote}
%     $^{\rm a}$Table footnote; $^{\rm b}$second table footnote.
%   \end{tabnote}
% \end{table}

% % Example of lists
% \subsection{Lists and Extracts} Here is an example of a numbered list:
% \begin{enumerate}
%   \item List entry number 1,
%   \item List entry number 2,
%   \item List entry number 3,\item List entry number 4, and
%   \item List entry number 5.
% \end{enumerate}

% Here is an example of a extract.
% \begin{extract}
%   This is an example text of quote or extract.
%   This is an example text of quote or extract.
% \end{extract}

% \subsection{Sidebars and Margin Notes}
% % Margin Note
% \begin{armarginnote}[]
%   \entry{Term A}{definition}
%   \entry{Term B}{definition}
%   \entry{Term C}{defintion}
% \end{armarginnote}

% \begin{textbox}[ht]\section{SIDEBARS}
%   Sidebar text goes here.
%   \subsection{Sidebar Second-Level Heading}
%   More text goes here.\subsubsection{Sidebar third-level heading}
%   Text goes here.\end{textbox}



% \subsection{Equations}
% % Example of a single-line equation
% \begin{equation}
%   a = b \ {\rm ((Single\ Equation\ Numbered))}
% \end{equation}
% %Example of multiple-line equation
% Equations can also be multiple lines as shown in Equations 2 and 3.
% \begin{eqnarray}
%   c = 0 \ {\rm ((Multiple\  Lines, \ Numbered))}\\
%   ac = 0 \ {\rm ((Multiple \ Lines, \ Numbered))}
% \end{eqnarray}

% Summary Points
%\begin{summary}[SUMMARY POINTS]
%  \begin{enumerate}
%    \item Summary point 1. These should be full sentences.
%    \item Summary point 2. These should be full sentences.
%    \item Summary point 3. These should be full sentences.
%    \item Summary point 4. These should be full sentences.
%  \end{enumerate}
%\end{summary}

% Future Issues
%\begin{issues}[FUTURE ISSUES]
%%  \begin{enumerate}
 %   \item Future issue 1. These should be full sentences.
 %   \item Future issue 2. These should be full sentences.
 %   \item Future issue 3. These should be full sentences.
 %   \item Future issue 4. These should be full sentences.
 % \end{enumerate}
%\end{issues}

%Disclosure
\section*{DISCLOSURE STATEMENT}
The authors are not aware of any affiliations, memberships, funding, or financial holdings that might be perceived as affecting the objectivity of this review.

% Acknowledgements
\section*{ACKNOWLEDGMENTS}
Acknowledgements, general annotations, funding.

% References
%
% Margin notes within bibliography
%\section*{LITERATURE\ CITED}

\bibliographystyle{ar-style2}
\bibliography{bib}

\end{document}
